\vspace{5 cm}
\textbf{\Huge{Resumen}}
\vspace{1.5 cm}

La rob�tica social es un campo de la rob�tica muy atractivo e interesante a cuya investigaci�n la comunidad rob�tica dedica cada vez m�s tiempo, esfuerzo y dinero. La interacci�n hombre-m�quina adquiere un valor esencial en este tipo de robots para que la comunicaci�n entre ambos sea intuitiva. En particular, la rob�tica asistencial se centra en c�mo pueden ayudar los robots a mejorar la calidad de vida de personas discapacitadas o enfermas. Una enfermedad que cada d�a afecta a m�s gente y se invierten m�s recursos en investigar sobre ella, es la enfermedad de Alzheimer.\\

El objetivo de este proyecto es emplear robots en terapias para enfermos de Alzheimer que sean eficaces y f�ciles de usar para el terapeuta. Para ellos hemos creado una aplicaci�n gr�fica que facilita este cometido. Esta herramienta, adem�s de una interfaz gr�fica, permite teleoperar al robot por medio un control remoto externo: el Wiimote.\\

El desarrollo de esta investigaci�n engloba, adem�s de la codificaci�n de la aplicaci�n, todos los aspectos relacionados con el uso y evaluaci�n de un robot en terapias de Alzheimer, y todas las herramientas utilizadas.\\

Esta investigaci�n ha sido realizada en el grupo Grupo de Rob�tica de la Universidad Rey Juan Carlos y se ha hecho en conjunto con un centro de investigaci�n especializado en enfermedades neurol�gicas, la Fundaci�n CIEN.